% Copyright (C) 2012 Shi.Zhan <g.shizhan.g@gmail.com>
%
% Permission is hereby granted, free of charge, to any person obtaining a copy of this software and associated documentation files (the "Software"), to deal in the Software without restriction, including without limitation the rights to use, copy, modify, merge, publish, distribute, sublicense, and/or sell copies of the Software, and to permit persons to whom the Software is furnished to do so, subject to the following conditions:
%
% The above copyright notice and this permission notice shall be included in all copies or substantial portions of the Software.
%
% THE SOFTWARE IS PROVIDED "AS IS", WITHOUT WARRANTY OF ANY KIND, EXPRESS OR IMPLIED, INCLUDING BUT NOT LIMITED TO THE WARRANTIES OF MERCHANTABILITY, FITNESS FOR A PARTICULAR PURPOSE AND NONINFRINGEMENT. IN NO EVENT SHALL THE AUTHORS OR COPYRIGHT HOLDERS BE LIABLE FOR ANY CLAIM, DAMAGES OR OTHER LIABILITY, WHETHER IN AN ACTION OF CONTRACT, TORT OR OTHERWISE, ARISING FROM, OUT OF OR IN CONNECTION WITH THE SOFTWARE OR THE USE OR OTHER DEALINGS IN THE SOFTWARE.
%
% 课程:人机交互技术及应用
% 班级:传播学1001班
% 课时:40学时,2012年秋季1~10周,每周一、三
% 地点:东九楼D212
% 主页:http://code.google.com/p/hci-course/
% 教师:施展 
% 单位:华中科技大学 武汉光电国家实验室
%
\documentclass{beamer}
\usepackage{fontspec,xunicode,xltxtra,beamerthemesplit}
%\usetheme{Hannover} % White background
\usetheme{Berkeley} % Blue background
\setmainfont[
	BoldFont={WenQuanYi Zen Hei},
	ItalicFont={WenQuanYi Micro Hei}
]{WenQuanYi Micro Hei}
\setsansfont[
	BoldFont={WenQuanYi Zen Hei},
	ItalicFont={WenQuanYi Micro Hei}
]{WenQuanYi Micro Hei}

% 中文环境自动换行
\XeTeXlinebreaklocale "zh"
\XeTeXlinebreakskip = 0pt plus 1pt

% 中文环境修正导航栏
\makeatletter
\def\beamer@linkspace#1{
	\begin{pgfpicture}{0pt}{-1.5pt}{#1}{5.5pt}
		\pgfsetfillopacity{0}
		\pgftext[x=0pt,y=-1.5pt]{.}
		\pgftext[x=#1,y=5.5pt]{.}
	\end{pgfpicture}
}
\makeatother

% diagrams
\usepackage{tikz}
\usetikzlibrary{arrows,shapes}

% full page image
\newcommand{\fullPageImage}[2]{
	{
		\usebackgroundtemplate{\includegraphics[width=\paperwidth, height=\paperheight]{#1}}
		\frame[plain]{#2}
	}
}

\title{人机交互技术}
\author{施展}
\institute{华中科技大学~武汉光电国家实验室}
\date{\today}
\titlegraphic{\includegraphics[width=2cm]{images/wnlo.jpg}}

\begin{document}

\begin{frame}
	\titlepage
\end{frame}

\begin{frame}
	\frametitle{内容提要}
	\tableofcontents
\end{frame}

\section{第七讲}
\begin{frame}
	\frametitle{第七讲 Web界面设计}
	\begin{itemize}
		\item 熟悉Web设计的原则及Web界面设计包含的元素。
		\item 掌握Web界面设计语言和技术,并灵活应用。
		% adjust to html5, introduce the most recent and promising techonology
	\end{itemize}
\end{frame}

\subsection{Web界面及相关概念}
\begin{frame}
	\frametitle{Web界面及相关概念}
	\beamertemplatetransparentcovereddynamicmedium
	\begin{itemize}[<+->]
		\item 万维网 World Wide Web, WWW
		\begin{itemize}
			\item 由高能核物理学家Tim Berners-Lee建立雏形: 一个能够整合各种资源、文件及多媒体的系统,让使用者方便地取得不同媒体的资料。 
			\item 建立在客户/服务器模型之上
			\begin{itemize}
				\item 超文本标记语言 Hypertext Markup Language, HTML
				\item 超文本传输协议 Hypertext Transport Protocols, HTTP
				\item 通过Internet把遍布世界各地的服务器连接起来,它能够提供各种Internet服务,具有一致用户界面的信息浏览功能。
			\end{itemize}
		\end{itemize}
	\end{itemize}
\end{frame}

\begin{frame}
	\frametitle{Web的发展趋势}
	\beamertemplatetransparentcovereddynamicmedium
	\begin{itemize}
		\item Web的爆炸性发展
		\begin{itemize}
			\item 各种媒体
			\item 各类交互
			\item 广泛渗透 
		\end{itemize}
		\pause
		\item W3C标准: XML
		\pause
		\item Web 2.0 \& 云
		\pause
		\item Web 3.0 \& Semantic Web
	\end{itemize}
\end{frame}

\subsection{Web界面设计原则}
\begin{frame}
	\frametitle{Web界面设计原则}
	\beamertemplatetransparentcovereddynamicmedium
	\begin{enumerate}[<+->]
		\item 以用户为中心
		\item 一致性
		\item 简洁与明确
		\item 体现特色
		\item 兼顾不同的浏览器
		\item 明确的导航设计
	\end{enumerate}
\end{frame}

\subsection{Web界面要素设计}
\begin{frame}
	\frametitle{Web界面要素设计}
	\beamertemplatetransparentcovereddynamicmedium
	\begin{enumerate}[<+->]
		\item Web界面规划
		\item 文化与语言
		\item 内容、风格与布局、色彩设计
		\item 文本设计
		\item 多媒体元素设计
	\end{enumerate}
\end{frame}

\subsection{Web界面基本技术}
\begin{frame}
	\frametitle{Web界面基本设计技术}
	\beamertemplatetransparentcovereddynamicmedium
	\begin{itemize}[<+->]
		\item 超文本标记语言HTML
		\item JavaApplet
		\item 客户端脚本语言JavaScript 
		\item 服务器端脚本语言
		\item 服务器端JavaScript
	\end{itemize}
\end{frame}

\subsection{Web界面新进展}
\begin{frame}
	\frametitle{Web界面新进展——HTML5}

\end{frame}

\section{小结}
\begin{frame}
	\frametitle{小结}
	\begin{itemize}
		\item 了解、认识Web界面及相关概念与关键技术
		\item 探讨Web界面设计原则、基本要素
	\end{itemize}
\end{frame}

\end{document}