% Copyright (C) 2012 Shi.Zhan <g.shizhan.g@gmail.com>
%
% Permission is hereby granted, free of charge, to any person obtaining a copy of this software and associated documentation files (the "Software"), to deal in the Software without restriction, including without limitation the rights to use, copy, modify, merge, publish, distribute, sublicense, and/or sell copies of the Software, and to permit persons to whom the Software is furnished to do so, subject to the following conditions:
%
% The above copyright notice and this permission notice shall be included in all copies or substantial portions of the Software.
%
% THE SOFTWARE IS PROVIDED "AS IS", WITHOUT WARRANTY OF ANY KIND, EXPRESS OR IMPLIED, INCLUDING BUT NOT LIMITED TO THE WARRANTIES OF MERCHANTABILITY, FITNESS FOR A PARTICULAR PURPOSE AND NONINFRINGEMENT. IN NO EVENT SHALL THE AUTHORS OR COPYRIGHT HOLDERS BE LIABLE FOR ANY CLAIM, DAMAGES OR OTHER LIABILITY, WHETHER IN AN ACTION OF CONTRACT, TORT OR OTHERWISE, ARISING FROM, OUT OF OR IN CONNECTION WITH THE SOFTWARE OR THE USE OR OTHER DEALINGS IN THE SOFTWARE.
%
% 课程:人机交互技术及应用
% 班级:传播学1001班
% 课时:40学时,2012年秋季1~10周,每周一、三
% 地点:东九楼D212
% 主页:http://code.google.com/p/hci-course/
% 教师:施展 
% 单位:华中科技大学 武汉光电国家实验室
%
\documentclass{beamer}
\usepackage{fontspec,xunicode,xltxtra,beamerthemesplit}
%\usetheme{Hannover} % White background
\usetheme{Berkeley} % Blue background
\setmainfont[
	BoldFont={WenQuanYi Zen Hei},
	ItalicFont={WenQuanYi Micro Hei}
]{WenQuanYi Micro Hei}
\setsansfont[
	BoldFont={WenQuanYi Zen Hei},
	ItalicFont={WenQuanYi Micro Hei}
]{WenQuanYi Micro Hei}

% 中文环境自动换行
\XeTeXlinebreaklocale "zh"
\XeTeXlinebreakskip = 0pt plus 1pt

% 中文环境修正导航栏
\makeatletter
\def\beamer@linkspace#1{
	\begin{pgfpicture}{0pt}{-1.5pt}{#1}{5.5pt}
		\pgfsetfillopacity{0}
		\pgftext[x=0pt,y=-1.5pt]{.}
		\pgftext[x=#1,y=5.5pt]{.}
	\end{pgfpicture}
}
\makeatother

\title{人机交互技术}
\author{施展}
\institute{华中科技大学~武汉光电国家实验室}
\date{\today}
\titlegraphic{\includegraphics[width=2cm]{images/wnlo.jpg}}

\begin{document}

\begin{frame}
	\titlepage
\end{frame}

\begin{frame}
	\frametitle{内容提要}
	\tableofcontents
\end{frame}

\section{第四讲}
\begin{frame}
	\frametitle{第四讲 交互技术}
	\begin{itemize}
		\item 人机交互输入模式
		\item 基本交互技术
		\item 图形交互技术
		\item 语音交互技术
		\item 笔交互技术
	\end{itemize}
\end{frame}

\subsection{人机交互输入模式}
\begin{frame}
	\frametitle{人机交互输入模式}
	\beamertemplatetransparentcovereddynamicmedium
	\begin{itemize}[<+->]
		\item 输入设备多种多样;
		\item 对一个应用程序而言,可以有多个输入设备,同一个设备又可能为多个任务服务;
		\item 要求对输入过程的处理要有合理的模式。
		\begin{itemize}
			\item 请求模式 (Request Mode)
			\item 采样模式 (Sample Mode)
			\item 事件模式 (Event Mode)
		\end{itemize}
	\end{itemize}
\end{frame}

\subsection{基本交互技术}
\begin{frame}
	\frametitle{基本交互技术}

\end{frame}

\subsection{图形交互技术}
\begin{frame}
	\frametitle{图形交互技术}

\end{frame}

\subsection{语音交互技术}
\begin{frame}
	\frametitle{语音交互技术}

\end{frame}

\subsection{笔交互技术}
\begin{frame}
	\frametitle{笔交互技术}

\end{frame}

\section{小结}
\begin{frame}
	\frametitle{小结}
	\begin{itemize}
		\item 理解人机交互输入模式
		\item 了解主要交互技术
	\end{itemize}
\end{frame}
 
\begin{frame}
	\frametitle{参考文献}
	\bibliographystyle{plain}
	\bibliography{hci}
\end{frame}

\end{document}